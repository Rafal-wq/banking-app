\documentclass[12pt,a4paper]{article}
\usepackage[utf8]{inputenc}
\usepackage[polish]{babel}
\usepackage[T1]{fontenc}
\usepackage{geometry}
\usepackage{graphicx}
\usepackage{amsmath}
\usepackage{amsfonts}
\usepackage{amssymb}
\usepackage{enumitem}
\usepackage{booktabs}
\usepackage{longtable}
\usepackage{array}
\usepackage{xcolor}
\usepackage{hyperref}
\usepackage{listings}
\usepackage{fancyhdr}

\geometry{margin=2.5cm}
\pagestyle{fancy}
\fancyhf{}
\fancyhead[L]{Aplikacja Bankowa - Opis funkcjonalny}
\fancyhead[R]{\thepage}

\title{\textbf{Aplikacja Bankowa}\\[0.5cm]\Large Opis funkcjonalny systemu}
\author{Dokumentacja techniczna}
\date{\today}

\begin{document}

    \maketitle
    \thispagestyle{empty}

    \newpage
    \tableofcontents
    \newpage

    \section{Wprowadzenie}

    Aplikacja Bankowa to nowoczesny system bankowości internetowej zbudowany przy użyciu technologii Laravel (backend) i React (frontend). System umożliwia użytkownikom zarządzanie kontami bankowymi, wykonywanie przelewów, śledzenie historii transakcji oraz wymianę walut. Aplikacja została zaprojektowana z myślą o bezpieczeństwie, skalowalności i łatwości użycia.

    \subsection{Cel systemu}

    Głównym celem systemu jest dostarczenie użytkownikom kompleksowej platformy do zarządzania finansami osobistymi, oferującej:
    \begin{itemize}
        \item Bezpieczny dostęp do kont bankowych
        \item Wykonywanie różnych typów transakcji finansowych
        \item Monitoring stanu finansów w czasie rzeczywistym
        \item Obsługę wielu walut z automatycznym przeliczaniem
        \item Przejrzystą historię wszystkich operacji
    \end{itemize}

    \subsection{Architektura systemu}

    System został zbudowany w architekturze Model-View-Controller (MVC) z wyraźnym podziałem na:
    \begin{itemize}
        \item \textbf{Backend} - Laravel 12 z API RESTful
        \item \textbf{Frontend} - React z Inertia.js dla seamless SPA experience
        \item \textbf{Baza danych} - MySQL z pełną obsługą transakcji ACID
        \item \textbf{Infrastruktura} - Docker dla spójnego środowiska deployment
    \end{itemize}

    \section{Moduły funkcjonalne systemu}

    \subsection{Moduł uwierzytelniania i autoryzacji}

    \subsubsection{Rejestracja użytkowników}
    System umożliwia rejestrację nowych użytkowników poprzez formularz zawierający:
    \begin{itemize}
        \item Imię i nazwisko (walidacja: wymagane, maksymalnie 255 znaków)
        \item Adres email (walidacja: unikalność, format email)
        \item Hasło (walidacja: minimum 6 znaków, potwierdzenie)
    \end{itemize}

    Po pomyślnej rejestracji użytkownik otrzymuje automatycznie wygenerowany token dostępowy i zostaje przekierowany do panelu głównego.

    \subsubsection{Logowanie}
    Funkcjonalność logowania obejmuje:
    \begin{itemize}
        \item Uwierzytelnianie za pomocą email i hasła
        \item Opcję "Zapamiętaj mnie" dla dłuższych sesji
        \item Mechanizm rate limiting (5 prób na minutę)
        \item Automatyczne generowanie tokenów API (Laravel Sanctum)
    \end{itemize}

    \subsubsection{Zarządzanie sesjami}
    \begin{itemize}
        \item Bezpieczne wylogowanie z invalidacją tokenów
        \item Automatyczne przekierowanie przy próbie dostępu do chronionych zasobów
        \item Obsługa CSRF protection dla wszystkich formulárzy
    \end{itemize}

    \subsection{Moduł zarządzania kontami bankowymi}

    \subsubsection{Tworzenie kont bankowych}
    Użytkownicy mogą tworzyć nieograniczoną liczbę kont bankowych z następującymi parametrami:
    \begin{itemize}
        \item \textbf{Nazwa konta} - dowolna nazwa opisowa (np. "Konto osobiste", "Oszczędności")
        \item \textbf{Waluta} - obsługiwane waluty: PLN, EUR, USD, GBP
        \item \textbf{Numer konta} - automatycznie generowany unikalny numer w formacie PLXXXXXXXXXXXXXXXX
        \item \textbf{Saldo początkowe} - domyślnie 0.00
    \end{itemize}

    \subsubsection{Program bonusu powitalnego}
    System oferuje atrakcyjny bonus powitalny dla nowych użytkowników:
    \begin{itemize}
        \item 1000 PLN dla pierwszego konta w złotówkach
        \item 220 EUR dla pierwszego konta w euro
        \item 250 USD dla pierwszego konta w dolarach
        \item 190 GBP dla pierwszego konta w funtach
    \end{itemize}

    Bonus jest automatycznie przyznawany tylko przy tworzeniu pierwszego konta przez użytkownika i rejestrowany jako transakcja systemowa.

    \subsubsection{Operacje na kontach}
    \begin{itemize}
        \item \textbf{Wpłaty} - możliwość doładowania konta dowolną kwotą
        \item \textbf{Wypłaty} - wypłacanie środków z kontrolą dostępnego salda
        \item \textbf{Dezaktywacja konta} - możliwość czasowego wyłączenia konta
        \item \textbf{Usuwanie konta} - możliwe tylko przy saldzie równym zero i braku oczekujących transakcji
    \end{itemize}

    \subsection{Moduł transakcji}

    \subsubsection{Przelewy wewnętrzne}
    System umożliwia wykonywanie przelewów między własnymi kontami użytkownika:
    \begin{itemize}
        \item Wybór konta źródłowego z listy dostępnych kont
        \item Wybór konta docelowego (różnego od źródłowego)
        \item Wprowadzenie kwoty z walidacją dostępnych środków
        \item Obowiązkowy tytuł przelewu
        \item Opcjonalny opis transakcji
    \end{itemize}

    \subsubsection{Przelewy zewnętrzne}
    Funkcjonalność przelewów na konta innych użytkowników:
    \begin{itemize}
        \item Wyszukiwanie konta odbiorcy po numerze konta
        \item Wyświetlanie danych odbiorcy (nazwa konta, właściciel)
        \item Wykonanie przelewu z pełną weryfikacją
        \item Ochrona przed przelewami na własne konta
    \end{itemize}

    \subsubsection{System statusów transakcji}
    Każda transakcja przechodzi przez następujące stany:
    \begin{itemize}
        \item \textbf{pending} - transakcja utworzona, oczekuje na przetworzenie
        \item \textbf{completed} - transakcja pomyślnie zrealizowana
        \item \textbf{failed} - transakcja odrzucona (np. niewystarczające środki)
    \end{itemize}

    \subsubsection{Mechanizm atomowości transakcji}
    System gwarantuje atomowość operacji finansowych poprzez:
    \begin{itemize}
        \item Użycie transakcji bazodanowych (DB::beginTransaction)
        \item Rollback w przypadku błędu na dowolnym etapie
        \item Walidację sald przed i po operacji
        \item Blokowanie kont podczas przetwarzania
    \end{itemize}

    \subsection{Moduł wymiany walut}

    \subsubsection{Obsługa wielu walut}
    System obsługuje cztery główne waluty z automatycznym przeliczaniem:
    \begin{itemize}
        \item \textbf{PLN} - Polski złoty (waluta bazowa)
        \item \textbf{EUR} - Euro (kurs: 1 EUR = 4.55 PLN)
        \item \textbf{USD} - Dolar amerykański (kurs: 1 USD = 4.00 PLN)
        \item \textbf{GBP} - Funt brytyjski (kurs: 1 GBP = 5.30 PLN)
    \end{itemize}

    \subsubsection{Automatyczne przewalutowanie}
    Przy przelewach między kontami w różnych walutach system automatycznie:
    \begin{itemize}
        \item Oblicza kurs wymiany na podstawie aktualnych kursów
        \item Przelicza kwotę na walutę docelową
        \item Zapisuje obie kwoty (źródłową i docelową) w bazie danych
        \item Dodaje informację o przewalutowaniu do opisu transakcji
        \item Wyświetla szczegóły przewalutowania w historii
    \end{itemize}

    \subsubsection{Wymiana walut}
    Dedykowana funkcjonalność wymiany walut pozwala na:
    \begin{itemize}
        \item Przeglądanie aktualnych kursów wszystkich par walutowych
        \item Kalkulację kwoty po przewalutowaniu przed wykonaniem operacji
        \item Wykonanie wymiany jako specjalnej transakcji między kontami
        \item Śledzenie wszystkich operacji walutowych w historii
    \end{itemize}

    \subsection{Moduł powiadomień}

    \subsubsection{Powiadomienia email}
    System automatycznie wysyła powiadomienia email o:
    \begin{itemize}
        \item Każdej wykonanej transakcji (wysyłka i odbiór)
        \item Różnych treściach dla transakcji wychodzących i przychodzących
        \item Szczegółowych informacjach o przewalutowaniu (jeśli dotyczy)
        \item Numerze referencyjnym transakcji
    \end{itemize}

    \subsubsection{Konfiguracja powiadomień}
    \begin{itemize}
        \item Możliwość wyłączenia powiadomień poprzez konfigurację MAIL\_ENABLED
        \item Mechanizm retry dla nieudanych wysyłek (maksymalnie 2 próby)
        \item Logowanie statusu wysyłki w systemie
        \item Toast notifications w interfejsie użytkownika
    \end{itemize}

    \subsection{Moduł raportowania i historii}

    \subsubsection{Historia transakcji}
    Kompleksowy system przeglądania historii transakcji obejmuje:
    \begin{itemize}
        \item Chronologiczną listę wszystkich transakcji użytkownika
        \item Filtrowanie po konkretnym koncie
        \item Oznaczenie kierunku transakcji (przychodząca/wychodząca)
        \item Wyświetlanie kwot w odpowiednich walutach
        \item Szczegółowe informacje o każdej transakcji
    \end{itemize}

    \subsubsection{Szczegóły transakcji}
    Każda transakcja zawiera pełne informacje:
    \begin{itemize}
        \item Data i czas wykonania
        \item Kwota źródłowa i docelowa (przy przewalutowaniu)
        \item Kurs wymiany (jeśli dotyczy)
        \item Dane kont uczestniczących
        \item Tytuł i opis operacji
        \item Unikalny numer referencyjny
        \item Status realizacji
    \end{itemize}

    \subsection{Moduł profilu użytkownika}

    \subsubsection{Zarządzanie danymi}
    Użytkownicy mogą aktualizować:
    \begin{itemize}
        \item Imię i nazwisko
        \item Adres email (z walidacją unikalności)
        \item Hasło (z wymaganą weryfikacją obecnego hasła)
    \end{itemize}

    \subsubsection{Bezpieczeństwo profilu}
    \begin{itemize}
        \item Wymaganie obecnego hasła przy zmianie danych
        \item Walidacja siły nowego hasła
        \item Automatyczne wylogowanie po zmianie krytycznych danych
    \end{itemize}

    \section{Bezpieczeństwo systemu}

    \subsection{Uwierzytelnianie i autoryzacja}
    \begin{itemize}
        \item \textbf{Laravel Sanctum} - nowoczesny system tokenów API
        \item \textbf{CSRF Protection} - ochrona przed atakami Cross-Site Request Forgery
        \item \textbf{Rate Limiting} - ograniczenie liczby prób logowania
        \item \textbf{Password Hashing} - bcrypt z salt dla wszystkich haseł
    \end{itemize}

    \subsection{Ochrona danych}
    \begin{itemize}
        \item \textbf{Walidacja danych} - wszystkie dane wejściowe są walidowane
        \item \textbf{SQL Injection Protection} - używanie Eloquent ORM
        \item \textbf{XSS Protection} - automatyczne escapowanie danych w widokach
        \item \textbf{Autoryzacja zasobów} - kontrola dostępu do kont i transakcji
    \end{itemize}

    \subsection{Integralność finansowa}
    \begin{itemize}
        \item \textbf{Transakcje ACID} - atomowość operacji finansowych
        \item \textbf{Walidacja sald} - kontrola dostępnych środków przed operacjami
        \item \textbf{Immutable History} - niemodyfikowalna historia transakcji
        \item \textbf{Audit Trail} - pełny ślad wszystkich operacji
    \end{itemize}

    \section{Interfejs użytkownika}

    \subsection{Responsive Design}
    Aplikacja oferuje w pełni responsywny interfejs:
    \begin{itemize}
        \item Automatyczne dostosowanie do urządzeń mobilnych
        \item Intuicyjna nawigacja na wszystkich rozdzielczościach
        \item Optymalizacja dla urządzeń dotykowych
        \item Czytelne formularze i tabele na małych ekranach
    \end{itemize}

    \subsection{User Experience}
    \begin{itemize}
        \item \textbf{Single Page Application} - płynne przejścia bez przeładowywania
        \item \textbf{Real-time Updates} - natychmiastowe odświeżanie sald
        \item \textbf{Loading States} - wskaźniki ładowania dla wszystkich operacji
        \item \textbf{Error Handling} - przyjazne komunikaty błędów
        \item \textbf{Toast Notifications} - dyskretne powiadomienia o operacjach
    \end{itemize}

    \subsection{Dostępność}
    \begin{itemize}
        \item Semantyczny HTML dla czytników ekranu
        \item Odpowiedni kontrast kolorów
        \item Keyboard navigation support
        \item ARIA labels dla interaktywnych elementów
    \end{itemize}

    \section{Integracje zewnętrzne}

    \subsection{Kursy walut}
    System może być zintegrowany z zewnętrznymi dostawcami kursów walut:
    \begin{itemize}
        \item ExchangeRate-API dla aktualnych kursów
        \item Alpha Vantage dla danych finansowych
        \item Automatyczne cache'owanie kursów (15 minut)
        \item Fallback na statyczne kursy przy braku połączenia
    \end{itemize}

    \subsection{Dane giełdowe}
    Strona główna wyświetla aktualne dane z głównych giełd:
    \begin{itemize}
        \item NYSE (S\&P 500)
        \item London Stock Exchange (FTSE 100)
        \item Tokyo Stock Exchange (Nikkei 225)
        \item Automatyczne odświeżanie co 15 minut
    \end{itemize}

    \section{Zarządzanie błędami i logowanie}

    \subsection{Obsługa błędów}
    \begin{itemize}
        \item Centralne logowanie wszystkich błędów aplikacji
        \item Graceful degradation przy błędach zewnętrznych serwisów
        \item Przyjazne komunikaty błędów dla użytkowników końcowych
        \item Automatyczny rollback transakcji przy błędach
    \end{itemize}

    \subsection{Monitoring}
    \begin{itemize}
        \item Logowanie wszystkich transakcji finansowych
        \item Monitoring prób nieautoryzowanego dostępu
        \item Śledzenie wydajności zapytań bazodanowych
        \item Alerty przy krytycznych błędach systemowych
    \end{itemize}

    \section{Wydajność i skalowalność}

    \subsection{Cache'owanie}
    \begin{itemize}
        \item Redis/Database cache dla często używanych danych
        \item Cache'owanie kont użytkownika (5 minut)
        \item Cache'owanie historii transakcji
        \item Automatyczne invalidowanie cache przy zmianach
    \end{itemize}

    \subsection{Optymalizacja}
    \begin{itemize}
        \item Lazy loading relacji w Eloquent
        \item Indeksowanie kolumn bazodanowych
        \item Kompresja zasobów statycznych
        \item CDN-ready asset management
    \end{itemize}

    \section{Deployment i infrastruktura}

    \subsection{Konteneryzacja}
    Aplikacja jest w pełni skonteneryzowana z użyciem Docker:
    \begin{itemize}
        \item Oddzielne kontenery dla aplikacji, bazy danych i cache
        \item Docker Compose dla łatwego zarządzania środowiskiem
        \item Wolumeny dla trwałego przechowywania danych
        \item Skrypty automatyzujące deployment
    \end{itemize}

    \subsection{Środowiska}
    \begin{itemize}
        \item \textbf{Development} - lokalny Docker z hot reload
        \item \textbf{Staging} - środowisko testowe z produkcyjnymi danymi
        \item \textbf{Production} - zabezpieczone środowisko produkcyjne
    \end{itemize}

    \section{Podsumowanie}

    Aplikacja Bankowa stanowi kompleksowe rozwiązanie dla bankowości internetowej, oferujące:

    \begin{itemize}
        \item Pełną funkcjonalność zarządzania kontami bankowymi
        \item Bezpieczne wykonywanie transakcji finansowych
        \item Zaawansowaną obsługę wielu walut
        \item Intuicyjny i responsywny interfejs użytkownika
        \item Wysokie standardy bezpieczeństwa
        \item Skalowalną architekturę
        \item Łatwość deployment i maintenance
    \end{itemize}

    System został zaprojektowany z myślą o przyszłej rozbudowie i może być łatwo rozszerzony o dodatkowe funkcjonalności, takie jak:
    \begin{itemize}
        \item Integracja z zewnętrznymi systemami płatniczymi
        \item Zaawansowane narzędzia analityczne
        \item Mobilna aplikacja companion
        \item API dla partnerów biznesowych
        \item Zaawansowane produkty finansowe (kredyty, lokaty)
    \end{itemize}

\end{document}
