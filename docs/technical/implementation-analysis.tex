\section{Krok 3: Wdrożone zagadnienia kwalifikacyjne}

W projekcie aplikacji bankowej zostało zaimplementowanych 18 kluczowych zagadnień technologicznych zgodnie z wymaganiami kwalifikacyjnymi. Poniżej przedstawiono szczegółowy opis każdego z zagadnień wraz z konkretными przykładami implementacji.

\subsection{1. Framework MVC}

Aplikacja została zbudowana w oparciu o framework Laravel 12, który implementuje wzorzec architektoniczny Model-View-Controller (MVC).

\textbf{Model:} Reprezentuje logikę biznesową i zarządzanie danymi:
\begin{itemize}
    \item \texttt{User.php} - model użytkownika z relacjami do kont bankowych
    \item \texttt{BankAccount.php} - model konta bankowego z metodami deposit/withdraw
    \item \texttt{Transaction.php} - model transakcji z logiką wykonywania przelewów
\end{itemize}

\textbf{View:} Interfejs użytkownika zrealizowany przez komponenty React z Inertia.js:
\begin{itemize}
    \item \texttt{Dashboard.jsx} - główny panel użytkownika
    \item \texttt{Accounts/Create.jsx} - formularz tworzenia konta
    \item \texttt{Transactions/Create.jsx} - formularz wykonywania przelewów
\end{itemize}

\textbf{Controller:} Kontrolery zarządzające logiką aplikacji:
\begin{itemize}
    \item \texttt{BankAccountController.php} - obsługa operacji na kontach
    \item \texttt{TransactionController.php} - zarządzanie transakcjami
    \item \texttt{AuthenticatedSessionController.php} - uwierzytelnianie
\end{itemize}

\subsection{2. Framework CSS}

Do stylizacji aplikacji wykorzystano framework \textbf{Tailwind CSS} w wersji 3.2.1, który oferuje podejście utility-first.

\textbf{Cechy wykorzystania:}
\begin{itemize}
    \item Responsive design z predefiniowanymi breakpointami
    \item Utility classes dla szybkiego prototypowania
    \item Customizacja poprzez \texttt{tailwind.config.js}
    \item Integracja z systemem budowania Vite
\end{itemize}

\textbf{Przykład implementacji:}
\begin{verbatim}
<div className="bg-white rounded-lg shadow-md p-6
                hover:shadow-lg transition-shadow">
    <h2 className="text-xl font-bold text-gray-800 mb-4">
        Twoje konta
    </h2>
</div>
\end{verbatim}

\subsection{3. Baza danych}

System wykorzystuje bazę danych \textbf{MySQL 8.0} z pełną obsługą transakcji ACID i relacji.

\textbf{Struktura bazy danych:}
\begin{itemize}
    \item \texttt{users} - dane użytkowników
    \item \texttt{bank\_accounts} - konta bankowe z saldami
    \item \texttt{transactions} - historia transakcji
    \item \texttt{personal\_access\_tokens} - tokeny uwierzytelniania
\end{itemize}

\textbf{Relacje:}
\begin{itemize}
    \item User $\rightarrow$ BankAccount (1:N)
    \item BankAccount $\rightarrow$ Transaction (1:N jako źródło i cel)
\end{itemize}

\textbf{Migracje Laravel:} System wykorzystuje migracje do zarządzania schematem bazy danych, zapewniając wersjonowanie struktury.

\subsection{4. Cache}

Implementacja mechanizmów cache'owania dla optymalizacji wydajności aplikacji.

\textbf{Cache bazy danych:} Wykorzystanie cache'u w tabeli \texttt{cache} MySQL:
\begin{verbatim}
$accounts = Cache::remember('user.'.$userId.'.accounts', 300,
    function () use ($user) {
        return $user->bankAccounts()->with(['transactions'])->get();
    });
\end{verbatim}

\textbf{Strategia cache'owania:}
\begin{itemize}
    \item Cache kont użytkownika (5 minut)
    \item Cache transakcji finansowych
    \item Cache kursów walut (15 minut)
    \item Cache danych giełdowych (15 minut)
\end{itemize}

\subsection{5. Dependency Manager}

Projekt wykorzystuje dwa główne managery zależności:

\textbf{Composer (PHP):} Zarządzanie pakietami PHP:
\begin{itemize}
    \item Laravel Framework 12.0
    \item Laravel Sanctum 4.0 (uwierzytelnianie)
    \item Inertia.js Laravel 2.0
    \item Tightenco Ziggy 2.0 (routing)
\end{itemize}

\textbf{NPM (JavaScript):} Zarządzanie pakietami JavaScript:
\begin{itemize}
    \item React 18.2.0
    \item Vite 6.2.0 (build tool)
    \item Tailwind CSS 3.2.1
    \item Axios 1.8.1 (HTTP client)
\end{itemize}

\subsection{6. HTML}

Semantyczny HTML generowany przez komponenty React z odpowiednimi znacznikami.

\textbf{Cechy implementacji:}
\begin{itemize}
    \item Semantyczne znaczniki (\texttt{<main>}, \texttt{<nav>}, \texttt{<section>})
    \item Accessibility przez ARIA labels
    \item Meta tagi dla SEO i viewport
    \item Formulary z odpowiednimi typami input
\end{itemize}

\textbf{Główny szablon:} \texttt{app.blade.php} z integracją Inertia.js:
\begin{verbatim}
<!DOCTYPE html>
<html lang="{{ str_replace('_', '-', app()->getLocale()) }}">
<head>
    <meta charset="utf-8">
    <meta name="viewport" content="width=device-width, initial-scale=1">
    <meta name="csrf-token" content="{{ csrf_token() }}">
    @viteReactRefresh
    @vite(['resources/js/app.jsx'])
</head>
<body>@inertia</body>
</html>
\end{verbatim}

\subsection{7. CSS}

Stylizacja realizowana poprzez Tailwind CSS z dodatkowymi customowymi komponentami.

\textbf{Organizacja stylów:}
\begin{itemize}
    \item \texttt{app.css} - główny plik stylów z dyrektywami Tailwind
    \item Komponenty React ze stylizacją inline przez className
    \item Responsive design dla wszystkich rozdzielczości
    \item Dark mode support (dostępny w Tailwind)
\end{itemize}

\textbf{Przykład wykorzystania:}
\begin{verbatim}
@tailwind base;
@tailwind components;
@tailwind utilities;
\end{verbatim}

\subsection{8. JavaScript}

Nowoczesny JavaScript ES6+ w architekturze komponentowej React.

\textbf{Główne funkcjonalności:}
\begin{itemize}
    \item Komponenty funkcyjne z React Hooks
    \item Asynchroniczne operacje z async/await
    \item Zarządzanie stanem przez useState/useEffect
    \item Event handling dla formularzy i interakcji
    \item Axios dla komunikacji HTTP
\end{itemize}

\textbf{Przykład komponentu:}
\begin{verbatim}
const [accounts, setAccounts] = useState([]);
const [loading, setLoading] = useState(true);

useEffect(() => {
    const fetchAccounts = async () => {
        const response = await axios.get('/api/bank-accounts');
        setAccounts(response.data.data);
        setLoading(false);
    };
    fetchAccounts();
}, []);
\end{verbatim}

\subsection{9. Routing}

System routingu oparty na Laravel z integracją Inertia.js.

\textbf{Routing backendowy (Laravel):}
\begin{itemize}
    \item \texttt{web.php} - trasy dla widoków Inertia
    \item \texttt{api.php} - RESTful API endpoints
    \item \texttt{auth.php} - trasy uwierzytelniania
    \item Middleware dla autoryzacji i CSRF
\end{itemize}

\textbf{Przykłady tras:}
\begin{verbatim}
// RESTful API routes
Route::apiResource('bank-accounts', BankAccountController::class);
Route::apiResource('transactions', TransactionController::class);

// Inertia routes
Route::get('/dashboard', function () {
    return Inertia::render('Dashboard');
})->middleware(['auth']);
\end{verbatim}

\textbf{Ziggy integration:} Udostępnianie tras Laravel w JavaScript przez pakiet Ziggy.

\subsection{10. ORM}

Wykorzystanie Eloquent ORM dla łatwego zarządzania danymi i relacjami.

\textbf{Modele z relacjami:}
\begin{verbatim}
// User model
public function bankAccounts(): HasMany
{
    return $this->hasMany(BankAccount::class);
}

// BankAccount model
public function outgoingTransactions(): HasMany
{
    return $this->hasMany(Transaction::class, 'from_account_id');
}

public function incomingTransactions(): HasMany
{
    return $this->hasMany(Transaction::class, 'to_account_id');
}
\end{verbatim}

\textbf{Query Builder:} Zaawansowane zapytania z eager loading:
\begin{verbatim}
$transactions = Transaction::where(function ($query) use ($accountIds) {
    $query->whereIn('from_account_id', $accountIds)
          ->orWhereIn('to_account_id', $accountIds);
})->with(['fromAccount', 'toAccount'])->get();
\end{verbatim}

\subsection{11. Uwierzytelnianie}

Bezpieczny system uwierzytelniania oparty na Laravel Sanctum.

\textbf{Mechanizmy uwierzytelniania:}
\begin{itemize}
    \item Session-based authentication dla SPA
    \item API Token authentication
    \item CSRF protection
    \item Rate limiting (5 prób na minutę)
    \item Password hashing z bcrypt
\end{itemize}

\textbf{Implementacja:}
\begin{verbatim}
// Login controller
public function store(LoginRequest $request)
{
    $request->authenticate();
    $request->session()->regenerate();

    $token = Auth::user()->createToken('api-token')->plainTextToken;

    return response()->json([
        'success' => true,
        'token' => $token,
        'user' => Auth::user()
    ]);
}
\end{verbatim}

\subsection{12. Mailing}

Zintegrowany system powiadomień email z wieloma próbami wysyłki.

\textbf{Klasy mailowe:}
\begin{itemize}
    \item \texttt{TransactionNotification} - powiadomienia o transakcjach
    \item \texttt{SimpleTestMail} - email testowy
    \item Szablony Blade dla formatowania HTML
\end{itemize}

\textbf{Funkcjonalności:}
\begin{verbatim}
// Wysyłka z retry mechanism
private function attemptEmailSending(Transaction $transaction): bool
{
    $maxAttempts = 2;

    for ($attempt = 1; $attempt <= $maxAttempts; $attempt++) {
        try {
            Mail::to($transaction->fromAccount->user->email)
                ->send(new TransactionNotification($transaction, true));
            return true;
        } catch (\Throwable $e) {
            Log::warning("Email attempt {$attempt} failed");
            sleep(1);
        }
    }
    return false;
}
\end{verbatim}

\subsection{13. Formularze}

Kompleksowe formularze z walidacją po stronie klienta i serwera.

\textbf{Rodzaje formularzy:}
\begin{itemize}
    \item Rejestracja i logowanie użytkowników
    \item Tworzenie kont bankowych
    \item Wykonywanie przelewów wewnętrznych i zewnętrznych
    \item Edycja profilu użytkownika
    \item Wymiana walut
\end{itemize}

\textbf{Walidacja:}
\begin{verbatim}
// Laravel validation
$request->validate([
    'name' => 'required|string|max:255',
    'currency' => ['required', 'string', Rule::in(['PLN', 'EUR', 'USD', 'GBP'])],
    'amount' => 'required|numeric|min:0.01',
]);

// React form handling
const { data, setData, post, processing, errors } = useForm({
    from_account_id: '',
    to_account_id: '',
    amount: '',
    title: '',
});
\end{verbatim}

\subsection{14. Asynchroniczne interakcje}

Implementacja asynchronicznych operacji dla płynnego UX.

\textbf{Frontend (React):}
\begin{itemize}
    \item useEffect dla asynchronicznego ładowania danych
    \item Loading states podczas oczekiwania na odpowiedź
    \item Error handling z toast notifications
    \item Optimistic updates dla lepszego UX
\end{itemize}

\textbf{Backend (Laravel):}
\begin{itemize}
    \item Asynchroniczne wysyłanie emaili
    \item Background processing dla transakcji
    \item Queue system dla długotrwałych operacji
\end{itemize}

\textbf{Przykład implementacji:}
\begin{verbatim}
const [loading, setLoading] = useState(false);

const handleSubmit = async (e) => {
    setLoading(true);
    try {
        const response = await axios.post('/api/transactions', data);
        showSuccessMessage(response.data.message);
    } catch (error) {
        showErrorMessage(error.response.data.message);
    } finally {
        setLoading(false);
    }
};
\end{verbatim}

\subsection{15. Konsumpcja API}

Integracja z zewnętrznymi API oraz własne API RESTful.

\textbf{Zewnętrzne API:}
\begin{itemize}
    \item \textbf{Alpha Vantage API} - dane giełdowe (S\&P 500, FTSE 100, Nikkei 225)
    \item \textbf{ExchangeRate-API} - aktualne kursy walut
    \item Fallback na statyczne dane przy braku połączenia
    \item Cache'owanie odpowiedzi (15 minut)
\end{itemize}

\textbf{Własne API:}
\begin{verbatim}
// FinancialDataService.php
public function getStockIndices()
{
    return Cache::remember('stock_indices', 900, function () {
        $spData = $this->fetchAlphaVantageGlobalQuote('SPY');
        $ftseData = $this->fetchAlphaVantageGlobalQuote('FTSE.LON');
        return ['nyse' => $spData, 'london' => $ftseData];
    });
}

// Frontend consumption
const response = await axios.get('/api/exchange-rates');
setExchangeRates(response.data.data);
\end{verbatim}

\subsection{16. RWD (Responsywny frontend)}

Pełna responsywność na wszystkich urządzeniach poprzez Tailwind CSS i React.

\textbf{Implementacja RWD:}
\begin{itemize}
    \item Mobile-first approach
    \item Flexbox i CSS Grid layouts
    \item Responsive breakpoints (sm, md, lg, xl)
    \item Touch-friendly UI elements
    \item Adaptive navigation menus
\end{itemize}

\textbf{Przykłady responsive design:}
\begin{verbatim}
<div className="grid grid-cols-1 md:grid-cols-2 lg:grid-cols-3 gap-6">
    <div className="p-4 bg-white rounded-lg shadow
                    hover:shadow-md transition-shadow">
        <h3 className="text-lg font-semibold mb-2">
            {account.name}
        </h3>
    </div>
</div>

// Navigation
<div className="hidden sm:ms-6 sm:flex sm:items-center">
    <Dropdown>...</Dropdown>
</div>
<div className="-me-2 flex items-center sm:hidden">
    <button>Menu</button>
</div>
\end{verbatim}

\subsection{17. Logger}

Kompleksowy system logowania akcji i zdarzeń w aplikacji.

\textbf{Rodzaje logów:}
\begin{itemize}
    \item Logi transakcji finansowych
    \item Logi błędów aplikacji
    \item Logi uwierzytelniania
    \item Logi wysyłki emaili
    \item Logi API calls
\end{itemize}

\textbf{Implementacja:}
\begin{verbatim}
// Transaction logging
Log::info('Transaction created', [
    'transaction_id' => $transaction->id,
    'user_id' => $user->id,
    'amount' => $transaction->amount,
    'from_account' => $transaction->from_account_id,
    'to_account' => $transaction->to_account_id
]);

// Error logging
Log::error('Transaction failed', [
    'error' => $e->getMessage(),
    'user_id' => $user->id,
    'trace' => $e->getTraceAsString()
]);

// Email logging
Log::info("Transaction email sent", [
    'transaction_id' => $transaction->id,
    'recipient' => $user->email,
    'type' => $isOutgoing ? 'outgoing' : 'incoming'
]);
\end{verbatim}

\subsection{18. Deployment}

Aplikacja przygotowana do deployment na różnych środowiskach.

\textbf{Digital Ocean VPS Deployment:}
\begin{itemize}
    \item Konfiguracja nginx + PHP-FPM
    \item MySQL na tym samym serwerze
    \item Environment variables przez .env
    \item Process management przez systemd
\end{itemize}

\textbf{Docker dla Development:}
\begin{itemize}
    \item \texttt{docker-compose.yml} z wieloma serwisami
    \item Konteneryzacja: app, nginx, mysql, node, phpmyadmin
    \item Volume mounting dla development
    \item Hot reload z Vite
\end{itemize}

\textbf{Scripts automatyzujące:}
\begin{itemize}
    \item \texttt{init.sh} / \texttt{init.ps1} - inicjalizacja środowiska
    \item \texttt{start.sh} / \texttt{start.cmd} - automatyczne wykrywanie OS
    \item \texttt{.do/deploy.sh} - skrypt post-deploy
\end{itemize}

\textbf{Environment Configuration:}
\begin{verbatim}
// .env.docker dla development
APP_ENV=local
APP_DEBUG=true
DB_HOST=mysql
DB_PORT=3306

// .do/app.yaml dla Digital Ocean
name: banking-app
services:
- name: web
  build_command: |
    npm install && npm run build
    composer install --optimize-autoloader --no-dev
  run_command: heroku-php-apache2 public/
\end{verbatim}

\textbf{CI/CD Ready:}
\begin{itemize}
    \item Struktura przygotowana pod GitHub Actions
    \item Automated testing poprzez PHPUnit
    \item Build optimization dla produkcji
    \item Database migrations w deployment pipeline
\end{itemize}

\section{Podsumowanie zagadnień kwalifikacyjnych}

Wszystkie 18 wymaganych zagadnień zostało pomyślnie zaimplementowanych w aplikacji bankowej, demonstrując:

\begin{itemize}
    \item \textbf{Architekturę MVC} poprzez Laravel + React + Inertia.js
    \item \textbf{Nowoczesne technologie frontend} (React, Tailwind CSS, Vite)
    \item \textbf{Solidne fundamenty backend} (Laravel, MySQL, Eloquent ORM)
    \item \textbf{Bezpieczeństwo} (Sanctum, CSRF, validation, logging)
    \item \textbf{Integracje zewnętrzne} (API finansowe, mailing)
    \item \textbf{Optymalizację} (cache, asynchroniczne operacje)
    \item \textbf{Deployment ready} (Docker, Digital Ocean, scripts)
\end{itemize}

Każde zagadnienie zostało zrealizowane zgodnie z najlepszymi praktykami branżowymi i zapewnia wysoką jakość, bezpieczeństwo oraz skalowalność rozwiązania.
