\documentclass[12pt,a4paper]{article}
\usepackage[utf8]{inputenc}
\usepackage[polish]{babel}
\usepackage[T1]{fontenc}
\usepackage{geometry}
\usepackage{graphicx}
\usepackage{amsmath}
\usepackage{amsfonts}
\usepackage{amssymb}
\usepackage{enumitem}
\usepackage{booktabs}
\usepackage{longtable}
\usepackage{array}
\usepackage{xcolor}
\usepackage{hyperref}
\usepackage{listings}
\usepackage{fancyhdr}

\geometry{margin=2.5cm}
\pagestyle{fancy}
\fancyhf{}
\fancyhead[L]{Aplikacja Bankowa - Przewodnik wdrożenia}
\fancyhead[R]{\thepage}

\title{\textbf{Aplikacja Bankowa}\\[0.5cm]\Large Przewodnik wdrożenia i uruchomienia systemu}
\author{Dokumentacja techniczna}
\date{\today}

% Konfiguracja listingów kodu
\lstset{
    basicstyle=\ttfamily\footnotesize,
    backgroundcolor=\color{gray!10},
    frame=single,
    breaklines=true,
    showstringspaces=false,
    numbers=left,
    numberstyle=\tiny\color{gray},
    captionpos=b
}

\begin{document}

    \maketitle
    \thispagestyle{empty}

    \newpage
    \tableofcontents
    \newpage

    \section{Wprowadzenie}

    Niniejszy dokument zawiera kompletną instrukcję wdrożenia i uruchomienia aplikacji bankowej w różnych środowiskach. System może być uruchamiany lokalnie w środowisku deweloperskim oraz zdalnie na serwerze produkcyjnym.

    \subsection{Wymagania systemowe}

    \textbf{Minimalne wymagania:}
    \begin{itemize}
        \item PHP 8.2 lub nowszy
        \item Node.js 18.0 lub nowszy
        \item MySQL 8.0 lub nowszy
        \item Composer 2.0 lub nowszy
        \item NPM 8.0 lub nowszy
        \item Git (do klonowania repozytorium)
    \end{itemize}

    \textbf{Wymagania dodatkowe dla środowiska deweloperskiego:}
    \begin{itemize}
        \item Docker i Docker Compose (opcjonalnie)
        \item Dostęp do portów: 8000 (Laravel), 5173 (Vite), 3306 (MySQL)
        \item Minimum 4GB RAM
        \item 2GB wolnego miejsca na dysku
    \end{itemize}

    \section{Uruchomienie lokalne}

    \subsection{Przygotowanie środowiska}

    \subsubsection{Klonowanie repozytorium}

    \begin{lstlisting}[language=bash, caption=Klonowanie projektu]
# Klonuj repozytorium
git clone [URL_REPOZYTORIUM]
cd banking-app

# Sprawdź strukturę projektu
ls -la
    \end{lstlisting}

    \subsubsection{Instalacja zależności}

    \textbf{Zależności PHP (Backend):}
    \begin{lstlisting}[language=bash, caption=Instalacja zależności PHP]
# Instalacja pakietów Composer
composer install

# Weryfikacja instalacji
composer --version
php --version
    \end{lstlisting}

    \textbf{Zależności JavaScript (Frontend):}
    \begin{lstlisting}[language=bash, caption=Instalacja zależności JavaScript]
# Instalacja pakietów NPM
npm install

# Weryfikacja instalacji
npm --version
node --version
    \end{lstlisting}

    \subsection{Konfiguracja środowiska}

    \subsubsection{Plik konfiguracyjny .env}

    \begin{lstlisting}[language=bash, caption=Przygotowanie pliku .env]
# Skopiuj przykładowy plik konfiguracyjny
cp .env.example .env

# Wygeneruj klucz aplikacji
php artisan key:generate
    \end{lstlisting}

    \textbf{Konfiguracja bazy danych w .env:}
    \begin{lstlisting}[caption=Przykładowa konfiguracja bazy danych]
APP_NAME=BankApp
APP_ENV=local
APP_KEY=base64:generated_key_here
APP_DEBUG=true
APP_URL=http://localhost:8000

DB_CONNECTION=mysql
DB_HOST=127.0.0.1
DB_PORT=3306
DB_DATABASE=banking_app
DB_USERNAME=root
DB_PASSWORD=your_password

BROADCAST_DRIVER=log
CACHE_DRIVER=database
FILESYSTEM_DISK=local
QUEUE_CONNECTION=database
SESSION_DRIVER=database
SESSION_LIFETIME=120

MAIL_MAILER=log
MAIL_ENABLED=true
MAIL_FROM_ADDRESS="noreply@bankapp.local"
MAIL_FROM_NAME="${APP_NAME}"
    \end{lstlisting}

    \subsubsection{Przygotowanie bazy danych}

    \textbf{Tworzenie bazy danych MySQL:}
    \begin{lstlisting}[language=sql, caption=SQL dla tworzenia bazy danych]
-- Zaloguj się do MySQL
mysql -u root -p

-- Utwórz bazę danych
CREATE DATABASE banking_app CHARACTER SET utf8mb4 COLLATE utf8mb4_unicode_ci;

-- Utwórz użytkownika (opcjonalnie)
CREATE USER 'bankapp_user'@'localhost' IDENTIFIED BY 'secure_password';
GRANT ALL PRIVILEGES ON banking_app.* TO 'bankapp_user'@'localhost';
FLUSH PRIVILEGES;

-- Wyjdź z MySQL
EXIT;
    \end{lstlisting}

    \textbf{Uruchomienie migracji:}
    \begin{lstlisting}[language=bash, caption=Migracje bazy danych]
# Wykonaj migracje
php artisan migrate

# Załaduj dane testowe (opcjonalnie)
php artisan db:seed

# Sprawdź status migracji
php artisan migrate:status
    \end{lstlisting}

    \subsection{Uruchomienie aplikacji}

    \subsubsection{Metoda standardowa (zalecana)}

    \textbf{Terminal 1 - Serwer Laravel:}
    \begin{lstlisting}[language=bash, caption=Uruchomienie serwera Laravel]
# Uruchom serwer deweloperski Laravel
php artisan serve

# Serwer będzie dostępny pod adresem:
# http://localhost:8000
    \end{lstlisting}

    \textbf{Terminal 2 - Serwer Vite (Frontend):}
    \begin{lstlisting}[language=bash, caption=Uruchomienie serwera Vite]
# W nowym terminalu uruchom Vite dev server
npm run dev

# Vite będzie dostępny pod adresem:
# http://localhost:5173
# Hot Module Replacement będzie aktywne
    \end{lstlisting}

    \subsubsection{Weryfikacja działania}

    \begin{itemize}
        \item Otwórz przeglądarkę i przejdź do \texttt{http://localhost:8000}
        \item Sprawdź, czy strona główna się ładuje
        \item Przetestuj rejestrację nowego użytkownika
        \item Sprawdź, czy hot reload działa podczas edycji plików React
    \end{itemize}

    \textbf{Testowanie podstawowych funkcjonalności:}
    \begin{lstlisting}[language=bash, caption=Testy połączenia z bazą danych]
# Test połączenia z bazą danych
php artisan tinker
>>> \DB::connection()->getPdo();
>>> exit

# Test endpointów API
curl http://localhost:8000/api/auth-check
    \end{lstlisting}

    \subsection{Alternatywne metody uruchomienia}

    \subsubsection{Docker (dla deweloperów preferujących konteneryzację)}

    \textbf{Automatyczne uruchomienie:}
    \begin{lstlisting}[language=bash, caption=Uruchomienie z Docker]
# Linux/macOS
./start.sh

# Windows
start.cmd

# Lub bezpośrednio przez init script
./init.sh  # Linux/macOS
init.ps1   # Windows PowerShell
    \end{lstlisting}

    \textbf{Ręczne uruchomienie Docker:}
    \begin{lstlisting}[language=bash, caption=Docker Compose]
# Uruchom wszystkie kontenery
docker-compose up -d

# Sprawdź status kontenerów
docker-compose ps

# Zainstaluj zależności w kontenerze
docker-compose exec app composer install
docker-compose exec app php artisan key:generate
docker-compose exec app php artisan migrate

# Zbuduj frontend
docker-compose run node npm install
docker-compose run node npm run build
    \end{lstlisting}

    \textbf{Porty używane przez Docker:}
    \begin{itemize}
        \item \texttt{8000} - Aplikacja główna (nginx + Laravel)
        \item \texttt{8080} - phpMyAdmin
        \item \texttt{5173} - Vite dev server (tylko w trybie dev)
        \item \texttt{3306} - MySQL (wewnętrzny)
    \end{itemize}

    \section{Uruchomienie zdalne}

    \subsection{Dostęp do aplikacji produkcyjnej}

    Aplikacja bankowa jest wdrożona i dostępna pod następującym adresem:

    \begin{center}
        \textbf{\Large \url{http://209.38.233.137/}}
    \end{center}

    \subsubsection{Informacje o środowisku produkcyjnym}

    \textbf{Specyfikacja serwera:}
    \begin{itemize}
        \item \textbf{Adres IP:} 209.38.233.137
        \item \textbf{Platforma:} Digital Ocean Droplet
        \item \textbf{System operacyjny:} Ubuntu 22.04 LTS
        \item \textbf{Serwer HTTP:} nginx 1.24
        \item \textbf{PHP:} 8.2-fpm
        \item \textbf{Baza danych:} MySQL 8.0
        \item \textbf{SSL:} HTTP (bez szyfrowania - środowisko deweloperskie)
    \end{itemize}

    \textbf{Architektura produkcyjna:}
    \begin{itemize}
        \item \textbf{Web Server:} nginx jako reverse proxy
        \item \textbf{Application Server:} PHP-FPM
        \item \textbf{Database:} MySQL na tym samym serwerze
        \item \textbf{Frontend:} Zbudowane zasoby statyczne (Vite build)
        \item \textbf{Session Storage:} Database-driven sessions
        \item \textbf{Cache:} Database cache (tabela cache)
    \end{itemize}

    \subsubsection{Konta testowe}

    Dla celów demonstracyjnych dostępne są następujące konta testowe:

    \textbf{Konto testowe \#1:}
    \begin{itemize}
        \item \textbf{Email:} test@example.com
        \item \textbf{Hasło:} password
        \item \textbf{Opis:} Konto z przykładowymi transakcjami i kontami bankowymi
    \end{itemize}

    \textbf{Możliwość rejestracji:}
    \begin{itemize}
        \item Można utworzyć własne konto przez formularz rejestracji
        \item Bonus powitalny 1000 PLN dla pierwszego konta
        \item Pełna funkcjonalność systemu bankowego
    \end{itemize}

    \subsection{Funkcjonalności dostępne zdalnie}

    \subsubsection{Pełny zakres funkcji}

    \begin{itemize}
        \item \textbf{Uwierzytelnianie:} Rejestracja i logowanie użytkowników
        \item \textbf{Zarządzanie kontami:} Tworzenie kont w różnych walutach
        \item \textbf{Przelewy wewnętrzne:} Transfery między własnymi kontami
        \item \textbf{Przelewy zewnętrzne:} Transfery do innych użytkowników
        \item \textbf{Wymiana walut:} Automatyczne przewalutowanie
        \item \textbf{Historia transakcji:} Pełna historia z filtrowaniem
        \item \textbf{Powiadomienia email:} Automatyczne powiadomienia o transakcjach
        \item \textbf{Responsywny interfejs:} Działa na wszystkich urządzeniach
    \end{itemize}

    \subsubsection{Dane finansowe w czasie rzeczywistym}

    \begin{itemize}
        \item \textbf{Kursy walut:} Aktualne kursy USD, EUR, GBP względem PLN
        \item \textbf{Indeksy giełdowe:} S\&P 500, FTSE 100, Nikkei 225
        \item \textbf{Aktualizacja:} Co 15 minut z zewnętrznych API
        \item \textbf{Fallback:} Statyczne dane przy braku połączenia z API
    \end{itemize}

    \subsection{Monitorowanie i diagnostyka}

    \subsubsection{Endpointy diagnostyczne}

    \textbf{Dostępne endpointy testowe:}
    \begin{lstlisting}[language=bash, caption=Testowanie endpointów zdalnych]
# Test połączenia z bazą danych
curl http://209.38.233.137/db-test

# Sprawdzenie konfiguracji
curl http://209.38.233.137/db-info

# Test sesji użytkownika
curl http://209.38.233.137/session-test

# Test konfiguracji poczty
curl http://209.38.233.137/mail-config
    \end{lstlisting}

    \textbf{Logi systemu:}
    \begin{itemize}
        \item Logi błędów aplikacji
        \item Logi transakcji finansowych
        \item Logi uwierzytelniania
        \item Logi wysyłki emaili
    \end{itemize}

    \section{Proces wdrażania (Deployment)}

    \subsection{Przygotowanie do wdrożenia}

    \subsubsection{Build zasobów produkcyjnych}

    \begin{lstlisting}[language=bash, caption=Budowanie zasobów dla produkcji]
# Zbuduj zasoby frontend dla produkcji
npm run build

# Zoptymalizuj autoloader Composera
composer install --optimize-autoloader --no-dev

# Zoptymalizuj Laravel
php artisan config:cache
php artisan route:cache
php artisan view:cache
php artisan optimize
    \end{lstlisting}

    \subsubsection{Konfiguracja środowiska produkcyjnego}

    \textbf{Plik .env dla produkcji:}
    \begin{lstlisting}[caption=Przykładowa konfiguracja produkcyjna]
APP_NAME=BankApp
APP_ENV=production
APP_KEY=base64:production_key_here
APP_DEBUG=false
APP_URL=http://209.38.233.137

DB_CONNECTION=mysql
DB_HOST=localhost
DB_PORT=3306
DB_DATABASE=banking_app_prod
DB_USERNAME=app_user
DB_PASSWORD=secure_production_password

CACHE_DRIVER=database
SESSION_DRIVER=database
QUEUE_CONNECTION=database

LOG_CHANNEL=daily
LOG_LEVEL=warning

MAIL_MAILER=smtp
MAIL_HOST=your-smtp-host
MAIL_PORT=587
MAIL_USERNAME=your-smtp-user
MAIL_PASSWORD=your-smtp-password
MAIL_ENCRYPTION=tls
    \end{lstlisting}

    \subsection{Digital Ocean Deployment}

    \subsubsection{Konfiguracja serwera}

    \textbf{nginx Configuration:}
    \begin{lstlisting}[language=nginx, caption=Konfiguracja nginx dla aplikacji]
server {
    listen 80;
    listen [::]:80;
    server_name 209.38.233.137;
    root /var/www/html/public;

    add_header X-Frame-Options "SAMEORIGIN";
    add_header X-Content-Type-Options "nosniff";

    index index.php;

    charset utf-8;

    location / {
        try_files $uri $uri/ /index.php?$query_string;
    }

    location = /favicon.ico { access_log off; log_not_found off; }
    location = /robots.txt  { access_log off; log_not_found off; }

    error_page 404 /index.php;

    location ~ \.php$ {
        fastcgi_pass unix:/var/run/php/php8.2-fpm.sock;
        fastcgi_param SCRIPT_FILENAME $realpath_root$fastcgi_script_name;
        include fastcgi_params;
    }

    location ~ /\.(?!well-known).* {
        deny all;
    }
}
    \end{lstlisting}

    \subsubsection{Skrypt wdrożenia}

    \begin{lstlisting}[language=bash, caption=Automatyczny skrypt deployment]
#!/bin/bash
# deploy.sh

echo "Starting deployment..."

# Pobierz najnowszy kod
git pull origin main

# Instaluj zależności
composer install --optimize-autoloader --no-dev
npm ci
npm run build

# Uruchom migracje
php artisan migrate --force

# Wyczyść i zoptymalizuj cache
php artisan config:clear
php artisan config:cache
php artisan route:cache
php artisan view:cache
php artisan optimize

# Ustaw uprawnienia
sudo chown -R www-data:www-data storage bootstrap/cache
sudo chmod -R 775 storage bootstrap/cache

# Restart serwisów
sudo systemctl reload nginx
sudo systemctl reload php8.2-fpm

echo "Deployment completed successfully!"
    \end{lstlisting}

    \subsection{Digital Ocean App Platform (Alternatywne)}

    \subsubsection{Konfiguracja app.yaml}

    Aplikacja zawiera także konfigurację do wdrożenia na Digital Ocean App Platform:

    \begin{lstlisting}[language=yaml, caption=Konfiguracja .do/app.yaml]
name: banking-app
services:
- name: web
  github:
    repo: [your-github-repo]
    branch: main
    deploy_on_push: true
  build_command: |
    npm install
    npm run build
    composer install --optimize-autoloader --no-dev
  run_command: heroku-php-apache2 public/
  envs:
  - key: APP_ENV
    value: production
  - key: APP_DEBUG
    value: false
  - key: APP_URL
    scope: RUN_AND_BUILD_TIME
    value: ${APP_URL}
  routes:
  - path: /
  health_check:
    http_path: /
    \end{lstlisting}

    \section{Rozwiązywanie problemów}

    \subsection{Problemy lokalne}

    \subsubsection{Błędy uruchomienia}

    \textbf{Problem: "php artisan serve" nie działa}
    \begin{lstlisting}[language=bash]
# Sprawdź wersję PHP
php --version

# Sprawdź, czy wszystkie rozszerzenia są zainstalowane
php -m | grep -E "(pdo|mysql|mbstring|openssl)"

# Regeneruj klucz aplikacji
php artisan key:generate

# Wyczyść cache
php artisan config:clear
php artisan cache:clear
    \end{lstlisting}

    \textbf{Problem: "npm run dev" kończy się błędem}
    \begin{lstlisting}[language=bash]
# Usuń node_modules i package-lock.json
rm -rf node_modules package-lock.json

# Reinstaluj pakiety
npm install

# Sprawdź wersję Node.js
node --version  # Powinno być >= 18.0

# Uruchom ponownie
npm run dev
    \end{lstlisting}

    \textbf{Problem: Błędy bazy danych}
    \begin{lstlisting}[language=bash]
# Sprawdź połączenie z MySQL
mysql -u root -p -e "SELECT 1"

# Sprawdź konfigurację w .env
cat .env | grep DB_

# Przetestuj połączenie Laravel
php artisan tinker
>>> \DB::connection()->getPdo();
    \end{lstlisting}

    \subsubsection{Błędy konfiguracji}

    \textbf{Problem: Strona wyświetla błąd 500}
    \begin{itemize}
        \item Sprawdź uprawnienia katalogów \texttt{storage/} i \texttt{bootstrap/cache/}
        \item Ustaw \texttt{APP\_DEBUG=true} w \texttt{.env} dla szczegółów błędu
        \item Sprawdź logi w \texttt{storage/logs/laravel.log}
    \end{itemize}

    \textbf{Problem: React komponenty się nie ładują}
    \begin{itemize}
        \item Sprawdź, czy Vite dev server działa na porcie 5173
        \item Sprawdź konfigurację CORS w \texttt{vite.config.js}
        \item Sprawdź console przeglądarki dla błędów JavaScript
    \end{itemize}

    \subsection{Problemy zdalne}

    \subsubsection{Diagnostyka serwera produkcyjnego}

    \textbf{Sprawdzanie statusu serwera:}
    \begin{lstlisting}[language=bash]
# Test dostępności
ping 209.38.233.137

# Test portu HTTP
telnet 209.38.233.137 80

# Test endpointu głównego
curl -I http://209.38.233.137/
    \end{lstlisting}

    \textbf{Typowe problemy i rozwiązania:}
    \begin{itemize}
        \item \textbf{502 Bad Gateway:} PHP-FPM nie działa lub błędna konfiguracja nginx
        \item \textbf{500 Internal Server Error:} Błąd aplikacji Laravel - sprawdź logi
        \item \textbf{Connection timeout:} Serwer może być niedostępny lub przeciążony
        \item \textbf{Brak stylów CSS:} Problem z budowaniem zasobów Vite
    \end{itemize}

    \section{Optymalizacja wydajności}

    \subsection{Optymalizacje lokalne}

    \begin{itemize}
        \item Użyj \texttt{php artisan serve} zamiast Apache/nginx dla developmentu
        \item Włącz opcaching PHP dla lepszej wydajności
        \item Skonfiguruj Redis dla cache (opcjonalnie)
        \item Użyj SSD dla lepszej wydajności I/O
    \end{itemize}

    \subsection{Optymalizacje produkcyjne}

    \begin{itemize}
        \item Włącz kompresję gzip w nginx
        \item Skonfiguruj cache headers dla zasobów statycznych
        \item Użyj CDN dla zasobów statycznych
        \item Optymalizuj obrazy i zasoby CSS/JS
        \item Skonfiguruj monitoring i alerty
    \end{itemize}

    \section{Bezpieczeństwo}

    \subsection{Zalecenia dla środowiska lokalnego}

    \begin{itemize}
        \item Nie używaj rzeczywistych danych finansowych w developmencie
        \item Użyj bezpiecznych haseł dla bazy danych
        \item Nie commituj plików \texttt{.env} do repozytorium
        \item Regularnie aktualizuj zależności
    \end{itemize}

    \subsection{Zalecenia dla środowiska produkcyjnego}

    \begin{itemize}
        \item Ustaw \texttt{APP\_DEBUG=false} w produkcji
        \item Skonfiguruj HTTPS/SSL dla rzeczywistej aplikacji
        \item Użyj silnych haseł dla wszystkich usług
        \item Skonfiguruj firewall i fail2ban
        \item Regularnie twórz kopie zapasowe bazy danych
        \item Monitoruj logi bezpieczeństwa
    \end{itemize}

    \section{Podsumowanie}

    Aplikacja bankowa może być uruchamiana w dwóch głównych trybach:

    \textbf{Środowisko lokalne:}
    \begin{itemize}
        \item Idealne do developmentu i testowania
        \item Wymaga \texttt{php artisan serve} + \texttt{npm run dev}
        \item Pełna funkcjonalność z hot reload
        \item Możliwość użycia Docker dla spójności środowiska
    \end{itemize}

    \textbf{Środowisko zdalne:}
    \begin{itemize}
        \item Dostępne pod adresem \url{http://209.38.233.137/}
        \item Pełna funkcjonalność bankowa
        \item Dane finansowe w czasie rzeczywistym
        \item Gotowe do demonstracji i testowania
    \end{itemize}

    Oba środowiska oferują kompletną funkcjonalność systemu bankowego, umożliwiając pełne testowanie i demonstrację wszystkich zaimplementowanych zagadnień kwalifikacyjnych.

\end{document}
