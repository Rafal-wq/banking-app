\subsection{External APIs}

\subsubsection{Financial Data Integration}
System aktywnie korzysta z zewnętrznych API:

\begin{itemize}
    \item \textbf{Alpha Vantage API} - dane giełdowe (S\&P 500, FTSE 100, Nikkei 225)
    \item \textbf{ExchangeRate-API} - aktualne kursy walut
    \item \textbf{Fallback Data} - statyczne dane awaryjne przy braku połączenia
    \item \textbf{Cache Strategy} - 15-minutowe cache'owanie odpowiedzi
\end{itemize}

\begin{lstlisting}[language=PHP, caption=Integracja z FinancialDataService]
class FinancialDataService
{
    public function getStockIndices()
    {
        return Cache::remember('stock_indices', 900, function () {
            // Alpha Vantage API calls
            $spData = $this->fetchAlphaVantageGlobalQuote('SPY');
            $ftseData = $this->fetchAlphaVantageGlobalQuote('FTSE.LON');
            // ...
        });
    }
}
\end{lstlisting}\documentclass[12pt,a4paper]{article}
\usepackage[utf8]{inputenc}
\usepackage[polish]{babel}
\usepackage[T1]{fontenc}
\usepackage{geometry}
\usepackage{graphicx}
\usepackage{amsmath}
\usepackage{amsfonts}
\usepackage{amssymb}
\usepackage{enumitem}
\usepackage{booktabs}
\usepackage{longtable}
\usepackage{array}
\usepackage{xcolor}
\usepackage{hyperref}
\usepackage{listings}
\usepackage{fancyhdr}
\usepackage{tikz}
\usetikzlibrary{positioning,arrows.meta,shapes.geometric}

\geometry{margin=2.5cm}
\pagestyle{fancy}
\fancyhf{}
\fancyhead[L]{Aplikacja Bankowa - Opis technologiczny}
\fancyhead[R]{\thepage}

\title{\textbf{Aplikacja Bankowa}\\[0.5cm]\Large Opis technologiczny systemu}
\author{Dokumentacja techniczna}
\date{\today}

% Konfiguracja listingów kodu
\lstset{
    basicstyle=\ttfamily\footnotesize,
    backgroundcolor=\color{gray!10},
    frame=single,
    breaklines=true,
    showstringspaces=false,
    numbers=left,
    numberstyle=\tiny\color{gray},
    captionpos=b
}

\begin{document}

    \maketitle
    \thispagestyle{empty}

    \newpage
    \tableofcontents
    \newpage

    \section{Wprowadzenie}

    Aplikacja Bankowa została zbudowana w oparciu o nowoczesny stos technologiczny, który łączy w sobie sprawdzone rozwiązania backendowe z dynamicznym frontendem. System wykorzystuje architekturę monolityczną z wyraźnym podziałem odpowiedzialności między warstwami aplikacji, zapewniając jednocześnie wysoką wydajność i łatwość utrzymania.

    \subsection{Wybór stosu technologicznego}

    Decyzja o wyborze konkretnych technologii została podjęta w oparciu o następujące kryteria:
    \begin{itemize}
        \item \textbf{Bezpieczeństwo} - krityczne znaczenie w aplikacjach finansowych
        \item \textbf{Skalowalność} - możliwość obsługi rosnącej liczby użytkowników
        \item \textbf{Łatwość development} - produktywność zespołu deweloperskiego
        \item \textbf{Ekosystem} - dostępność bibliotek i wsparcia społeczności
        \item \textbf{Performance} - szybkość działania aplikacji
    \end{itemize}

    \section{Architektura systemu}

    \subsection{Wzorzec architektoniczny}

    System został zbudowany w oparciu o architekturę \textbf{Modern Monolith} z następującymi charakterystykami:

    \begin{itemize}
        \item \textbf{Single Codebase} - jedna baza kodu dla całej aplikacji
        \item \textbf{Layered Architecture} - wyraźny podział na warstwy (prezentacji, logiki biznesowej, danych)
        \item \textbf{SPA Frontend} - Single Page Application z React
        \item \textbf{API-first approach} - RESTful API jako rdzeń komunikacji
        \item \textbf{Database-centric} - relacyjna baza danych jako źródło prawdy
        \item \textbf{Traditional Deployment} - nginx + Digital Ocean Droplet (bez konteneryzacji w produkcji)
    \end{itemize}

    \subsection{Diagram architektury}

    \begin{center}
        \begin{tikzpicture}[
            node distance=1.5cm,
            every node/.style={align=center},
            box/.style={rectangle, draw, fill=blue!10, minimum width=3cm, minimum height=1cm},
            connector/.style={->, thick}
        ]

% Frontend Layer
            \node[box] (react) {React + Inertia.js\\Frontend};
            \node[box, below=of react] (inertia) {Inertia.js\\Bridge};

% Backend Layer
            \node[box, below=of inertia] (laravel) {Laravel 12\\Backend};
            \node[box, left=of laravel] (auth) {Sanctum\\Authentication};
            \node[box, right=of laravel] (api) {RESTful\\API};

% Data Layer
            \node[box, below=of laravel] (eloquent) {Eloquent\\ORM};
            \node[box, below=of eloquent] (mysql) {MySQL 8.0\\Database};

% Infrastructure
            \node[box, left=of mysql] (docker) {Docker\\Containers};
            \node[box, right=of mysql] (vite) {Vite\\Build Tool};

% Connections
            \draw[connector] (react) -- (inertia);
            \draw[connector] (inertia) -- (laravel);
            \draw[connector] (laravel) -- (auth);
            \draw[connector] (laravel) -- (api);
            \draw[connector] (laravel) -- (eloquent);
            \draw[connector] (eloquent) -- (mysql);
            \draw[connector] (docker) -- (mysql);
            \draw[connector] (vite) -- (react);

        \end{tikzpicture}
    \end{center}

    \section{Stack technologiczny}

    \subsection{Specyfikacja wersji}

    System wykorzystuje następujące wersje technologii:

    \begin{table}[h]
        \centering
        \begin{tabular}{|l|l|l|}
            \hline
            \textbf{Technologia} & \textbf{Wersja} & \textbf{Opis} \\
            \hline
            PHP & 8.2 & Język programowania backend \\
            Laravel & 12.0 & Framework PHP \\
            Node.js & 18+ & Runtime JavaScript (wymagany przez Vite) \\
            React & 18.2.0 & Biblioteka frontend \\
            Inertia.js & 2.0.0 & Most Laravel-React \\
            MySQL & 8.0 & System bazy danych \\
            Vite & 6.2.0 & Build tool \\
            Tailwind CSS & 3.2.1 & Framework CSS \\
            nginx & latest & Serwer HTTP (produkcja) \\
            \hline
        \end{tabular}
        \caption{Wersje wykorzystywanych technologii}
    \end{table}

    \subsection{Backend - Laravel Framework}

    \subsubsection{Laravel 12}
    Laravel 12 stanowi rdzeń aplikacji backendowej, oferując:

    \begin{itemize}
        \item \textbf{Eloquent ORM} - zaawansowany mapper obiektowo-relacyjny
        \item \textbf{Artisan CLI} - narzędzie command-line do zarządzania aplikacją
        \item \textbf{Blade Templating} - system szablonów (używany minimalnie)
        \item \textbf{Routing System} - elastyczny system routingu
        \item \textbf{Middleware Stack} - przetwarzanie żądań HTTP
        \item \textbf{Service Container} - dependency injection container
    \end{itemize}

    \subsubsection{Kluczowe komponenty Laravel}

    \textbf{Models (Modele danych):}
    \begin{lstlisting}[language=PHP, caption=Przykład modelu User]
class User extends Authenticatable
{
    use HasApiTokens, HasFactory, Notifiable;

    protected $fillable = ['name', 'email', 'password'];

    public function bankAccounts(): HasMany
    {
        return $this->hasMany(BankAccount::class);
    }
}
    \end{lstlisting}

    \textbf{Controllers (Kontrolery):}
    \begin{itemize}
        \item \texttt{BankAccountController} - zarządzanie kontami
        \item \texttt{TransactionController} - obsługa transakcji
        \item \texttt{AuthenticationController} - uwierzytelnianie
    \end{itemize}

    \textbf{Middleware:}
    \begin{itemize}
        \item \texttt{auth:sanctum} - autoryzacja API
        \item \texttt{cors} - Cross-Origin Resource Sharing
        \item \texttt{csrf} - ochrona przed CSRF
        \item \texttt{throttle} - rate limiting
    \end{itemize}

    \subsection{Frontend - React Ecosystem}

    \subsubsection{React 18}
    Biblioteka React służy do budowy interfejsu użytkownika:

    \begin{itemize}
        \item \textbf{Functional Components} - komponenty funkcyjne z hooks
        \item \textbf{React Hooks} - useState, useEffect, custom hooks
        \item \textbf{JSX Syntax} - declaratywny sposób opisywania UI
        \item \textbf{Component Composition} - kompozycja komponentów
        \item \textbf{Event Handling} - obsługa zdarzeń użytkownika
    \end{itemize}

    \subsubsection{Inertia.js - The Modern Monolith}
    Inertia.js pełni rolę mostka między Laravel a React:

    \begin{itemize}
        \item \textbf{Server-side Routing} - wykorzystanie routingu Laravel
        \item \textbf{No API Required} - brak konieczności budowy oddzielnego API
        \item \textbf{SPA Experience} - płynne przejścia bez reload strony
        \item \textbf{Props Hydration} - automatyczne przekazywanie danych z backend
        \item \textbf{Form Handling} - uproszczona obsługa formularzy
    \end{itemize}

    \begin{lstlisting}[language=JavaScript, caption=Przykład komponentu React z Inertia]
import { Link, useForm } from '@inertiajs/react';

export default function CreateAccount() {
    const { data, setData, post, processing } = useForm({
        name: '',
        currency: 'PLN'
    });

    const handleSubmit = (e) => {
        e.preventDefault();
        post('/api/bank-accounts');
    };

    return (
        <form onSubmit={handleSubmit}>
            {/* Formularz tworzenia konta */}
        </form>
    );
}
    \end{lstlisting}

    \subsection{Stylizacja - Tailwind CSS}

    \subsubsection{Utility-first CSS Framework}
    Tailwind CSS zapewnia:

    \begin{itemize}
        \item \textbf{Utility Classes} - małe, pojedyncze klasy CSS
        \item \textbf{Responsive Design} - wbudowane media queries
        \item \textbf{Dark Mode Support} - obsługa trybu ciemnego
        \item \textbf{Customization} - łatwa konfiguracja poprzez tailwind.config.js
        \item \textbf{JIT Compilation} - Just-In-Time kompilacja CSS
    \end{itemize}

    \subsubsection{Headless UI Components}
    Dodatkowo wykorzystywane są komponenty Headless UI:
    \begin{itemize}
        \item \texttt{Dialog} - modale i okna dialogowe
        \item \texttt{Transition} - animacje przejść
        \item \texttt{Menu} - rozwijane menu
    \end{itemize}

    \subsection{Build Tools - Vite}

    \subsubsection{Next Generation Frontend Tooling}
    Vite obsługuje:

    \begin{itemize}
        \item \textbf{Lightning Fast HMR} - Hot Module Replacement
        \item \textbf{ES Modules} - natywne wsparcie dla ES modules
        \item \textbf{TypeScript Support} - wsparcie dla TypeScript
        \item \textbf{Asset Processing} - optymalizacja zasobów
        \item \textbf{Code Splitting} - automatyczne dzielenie kodu
    \end{itemize}

    \begin{lstlisting}[language=JavaScript, caption=Konfiguracja Vite]
export default defineConfig({
    plugins: [
        laravel({
            input: ['resources/js/app.jsx'],
            refresh: true,
        }),
        react(),
    ],
    server: {
        host: '0.0.0.0',
        port: 5173,
        hmr: { host: 'localhost' }
    }
});
    \end{lstlisting}

    \section{Baza danych}

    \subsection{MySQL 8.0}

    \subsubsection{Charakterystyka}
    MySQL 8.0 oferuje:

    \begin{itemize}
        \item \textbf{ACID Compliance} - atomowość, spójność, izolacja, trwałość
        \item \textbf{InnoDB Engine} - domyślny engine z obsługą transakcji
        \item \textbf{JSON Support} - natywne wsparcie dla dokumentów JSON
        \item \textbf{Window Functions} - zaawansowane funkcje analityczne
        \item \textbf{Performance Schema} - monitoring wydajności
    \end{itemize}

    \subsubsection{Schema bazy danych}

    \textbf{Główne tabele:}
    \begin{itemize}
        \item \texttt{users} - dane użytkowników
        \item \texttt{bank\_accounts} - konta bankowe
        \item \texttt{transactions} - historia transakcji
        \item \texttt{personal\_access\_tokens} - tokeny Sanctum
    \end{itemize}

    \textbf{Relacje:}
    \begin{itemize}
        \item User $\rightarrow$ BankAccount (1:N)
        \item BankAccount $\rightarrow$ Transaction (1:N jako źródło)
        \item BankAccount $\rightarrow$ Transaction (1:N jako cel)
    \end{itemize}

    \subsection{Migracje Laravel}

    System wykorzystuje migracje do zarządzania schematem:

    \begin{lstlisting}[language=PHP, caption=Przykład migracji]
Schema::create('bank_accounts', function (Blueprint $table) {
    $table->id();
    $table->foreignId('user_id')->constrained()->onDelete('cascade');
    $table->string('account_number')->unique();
    $table->string('name');
    $table->decimal('balance', 10, 2)->default(0);
    $table->string('currency')->default('PLN');
    $table->boolean('is_active')->default(true);
    $table->timestamps();
});
    \end{lstlisting}

    \section{Uwierzytelnianie i autoryzacja}

    \subsection{Laravel Sanctum}

    \subsubsection{API Token Authentication}
    Sanctum zapewnia:

    \begin{itemize}
        \item \textbf{Personal Access Tokens} - tokeny dla SPA
        \item \textbf{API Token Authentication} - uwierzytelnianie API
        \item \textbf{Cookie Authentication} - dla first-party apps
        \item \textbf{CSRF Protection} - ochrona przed CSRF w SPA
    \end{itemize}

    \subsubsection{Proces uwierzytelniania}

    \begin{enumerate}
        \item Użytkownik loguje się przez formularz
        \item Laravel weryfikuje dane i tworzy sesję
        \item Sanctum generuje token API
        \item Token przechowywany w localStorage
        \item Każde żądanie API zawiera token w nagłówku Authorization
    \end{enumerate}

    \subsection{Middleware Security Stack}

    \begin{itemize}
        \item \textbf{VerifyCsrfToken} - walidacja tokenów CSRF
        \item \textbf{EncryptCookies} - szyfrowanie cookies
        \item \textbf{TrimStrings} - normalizacja danych wejściowych
        \item \textbf{ValidatePostSize} - walidacja rozmiaru żądań
    \end{itemize}

    \section{API Design}

    \subsection{RESTful Architecture}

    System implementuje zasady REST:

    \begin{table}[h]
        \centering
        \begin{tabular}{|l|l|l|}
            \hline
            \textbf{Method} & \textbf{Endpoint} & \textbf{Opis} \\
            \hline
            GET & /api/bank-accounts & Lista kont użytkownika \\
            POST & /api/bank-accounts & Utworzenie nowego konta \\
            GET & /api/bank-accounts/\{id\} & Szczegóły konta \\
            PATCH & /api/bank-accounts/\{id\} & Aktualizacja konta \\
            DELETE & /api/bank-accounts/\{id\} & Usunięcie konta \\
            \hline
            POST & /api/transactions & Wykonanie transakcji \\
            GET & /api/transactions & Historia transakcji \\
            GET & /api/transactions/\{id\} & Szczegóły transakcji \\
            \hline
        \end{tabular}
        \caption{Główne endpointy API}
    \end{table}

    \subsection{Response Format}

    Standardowy format odpowiedzi API:

    \begin{lstlisting}[language=JSON, caption=Format odpowiedzi API]
{
    "success": true,
    "data": {
        "id": 1,
        "name": "Konto osobiste",
        "balance": 1000.00,
        "currency": "PLN"
    },
    "message": "Operation completed successfully"
}
    \end{lstlisting}

    \section{Infrastruktura i deployment}

    \subsection{Digital Ocean Droplet Deployment}

    \subsubsection{Traditional VPS Architecture}
    System deployowany jest na tradycyjnym VPS:

    \begin{itemize}
        \item \textbf{Digital Ocean Droplet} - Virtual Private Server
        \item \textbf{nginx} - serwer HTTP i reverse proxy
        \item \textbf{PHP-FPM} - FastCGI Process Manager dla PHP
        \item \textbf{MySQL} - baza danych na tym samym serwerze
        \item \textbf{No Docker in Production} - bezpośrednia instalacja na OS
        \item \textbf{HTTP Only} - brak certyfikatu SSL/HTTPS
    \end{itemize}

    \subsubsection{nginx Configuration}

    \begin{lstlisting}[language=nginx, caption=Przykład konfiguracji nginx]
server {
    listen 80;
    server_name your-domain.com;
    root /var/www/html/public;

    index index.php index.html;

    location / {
        try_files $uri $uri/ /index.php?$query_string;
    }

    location ~ \.php$ {
        fastcgi_pass unix:/var/run/php/php8.2-fpm.sock;
        fastcgi_index index.php;
        include fastcgi_params;
    }
}
    \end{lstlisting}

    \subsection{Docker dla Development Only}

    \subsubsection{Development Environment}
    Docker używany jest wyłącznie do lokalnego developmentu:

    \begin{itemize}
        \item \textbf{app} - PHP-FPM + Laravel aplikacja
        \item \textbf{nginx} - serwer webowy i reverse proxy
        \item \textbf{mysql} - baza danych MySQL
        \item \textbf{node} - budowanie zasobów frontendowych
        \item \textbf{phpmyadmin} - zarządzanie bazą danych
        \item \textbf{vite} - serwer deweloperski z HMR
    \end{itemize}

    \subsection{Environment Configuration}

    \subsubsection{Environment Variables}
    Konfiguracja przez zmienne środowiskowe:

    \begin{itemize}
        \item \texttt{APP\_ENV} - środowisko aplikacji (local/production)
        \item \texttt{APP\_DEBUG} - tryb debugowania
        \item \texttt{DB\_*} - konfiguracja bazy danych
        \item \texttt{MAIL\_*} - konfiguracja poczty
        \item \texttt{CACHE\_DRIVER} - driver cache'owania
    \end{itemize}

    \subsection{Digital Ocean App Platform}

    \subsubsection{VPS Deployment}
    Aplikacja wdrożona na Digital Ocean Droplet:

    \begin{itemize}
        \item \textbf{Manual Deployment} - tradycyjne wdrażanie na VPS
        \item \textbf{Direct Installation} - bezpośrednia instalacja na Ubuntu/Debian
        \item \textbf{File Transfer} - deployment przez git pull lub scp
        \item \textbf{Process Management} - zarządzanie procesami przez systemd
        \item \textbf{No Load Balancer} - single server architecture
        \item \textbf{HTTP Protocol} - brak szyfrowania SSL/TLS
    \end{itemize}

    \section{Performance i optymalizacja}

    \subsection{Cache Strategy}

    \subsubsection{Database Cache Only}
    System implementuje uproszczoną strategię cache'owania opartą wyłącznie na bazie danych:

    \begin{itemize}
        \item \textbf{Database Cache} - cache zapytań i danych aplikacji w tabeli cache
        \item \textbf{Session Cache} - przechowywanie sesji w bazie danych
        \item \textbf{No Redis} - świadoma decyzja o niekorzystaniu z Redis dla uproszczenia
        \item \textbf{Application Level} - cache na poziomie aplikacji Laravel
    \end{itemize}

    \begin{lstlisting}[language=PHP, caption=Przykład cache'owania]
$accounts = Cache::remember('user.'.$userId.'.accounts', 300, function () use ($user) {
    return $user->bankAccounts()->with(['transactions'])->get();
});
    \end{lstlisting}

    \subsection{Database Optimization}

    \begin{itemize}
        \item \textbf{Indexing} - indeksy na kluczowych kolumnach
        \item \textbf{Query Optimization} - optymalizacja zapytań Eloquent
        \item \textbf{Lazy Loading} - ładowanie relacji na żądanie
        \item \textbf{Connection Pooling} - pooling połączeń bazodanowych
    \end{itemize}

    \subsection{Frontend Optimization}

    \begin{itemize}
        \item \textbf{Code Splitting} - dzielenie kodu na chunki
        \item \textbf{Tree Shaking} - usuwanie nieużywanego kodu
        \item \textbf{Asset Minification} - minifikacja CSS/JS
        \item \textbf{Image Optimization} - optymalizacja obrazów
    \end{itemize}

    \section{Monitoring i logowanie}

    \subsection{Application Logging}

    \subsubsection{Structured Logging}
    System logowania obejmuje:

    \begin{itemize}
        \item \textbf{Error Logs} - błędy aplikacji
        \item \textbf{Access Logs} - logi dostępu HTTP
        \item \textbf{Transaction Logs} - logi operacji finansowych
        \item \textbf{Security Logs} - logi bezpieczeństwa
    \end{itemize}

    \begin{lstlisting}[language=PHP, caption=Przykład logowania]
Log::info('Transaction created', [
    'transaction_id' => $transaction->id,
    'user_id' => $user->id,
    'amount' => $transaction->amount,
    'from_account' => $transaction->from_account_id,
    'to_account' => $transaction->to_account_id
]);
    \end{lstlisting}

    \subsection{Error Handling}

    \begin{itemize}
        \item \textbf{Exception Handling} - centralne przechwytywanie błędów
        \item \textbf{Graceful Degradation} - elegancka obsługa awarii
        \item \textbf{User-friendly Messages} - przyjazne komunikaty błędów
        \item \textbf{Debug Information} - szczegółowe informacje w trybie dev
    \end{itemize}

    \section{Security Considerations}

    \subsection{Data Protection}

    \begin{itemize}
        \item \textbf{Database Encryption} - szyfrowanie wrażliwych danych w bazie
        \item \textbf{Password Hashing} - bcrypt dla haseł użytkowników
        \item \textbf{Token Security} - bezpieczne tokeny API Sanctum
        \item \textbf{HTTP Only} - komunikacja bez szyfrowania SSL (development/internal use)
        \item \textbf{Environment Variables} - wrażliwe dane w plikach .env
    \end{itemize}

    \subsection{Input Validation}

    \begin{itemize}
        \item \textbf{Form Requests} - walidacja danych wejściowych
        \item \textbf{Eloquent Mass Assignment} - ochrona przed mass assignment
        \item \textbf{SQL Injection Protection} - parametryzowane zapytania
        \item \textbf{XSS Protection} - automatyczne escapowanie danych
    \end{itemize}

    \section{Development Workflow}

    \subsection{Local Development}

    \subsubsection{Docker Development Environment}
    Środowisko deweloperskie oparte na Docker (różne od produkcji):

    \begin{enumerate}
        \item Clone repozytorium
        \item Uruchomienie \texttt{./start.sh} lub \texttt{start.cmd}
        \item Automatyczna konfiguracja kontenerów Docker
        \item Hot reload dla development z Vite
        \item Dostęp przez http://localhost:8000
    \end{enumerate}

    \subsubsection{Production Deployment}
    Deployment na Digital Ocean Droplet:

    \begin{enumerate}
        \item Konfiguracja VPS z nginx + PHP-FPM + MySQL
        \item Git pull lub transfer plików
        \item Composer install na serwerze
        \item npm run build dla zasobów frontend
        \item Migracje bazy danych
        \item Restart nginx/php-fpm
    \end{enumerate}

    \subsection{Code Quality}

    \begin{itemize}
        \item \textbf{PSR Standards} - przestrzeganie standardów PHP
        \item \textbf{ESLint/Prettier} - formatowanie kodu JavaScript
        \item \textbf{Git Hooks} - automatyczne sprawdzenia przed commit
    \end{itemize}

    \section{Podsumowanie technologiczne}

    Aplikacja Bankowa została zbudowana w oparciu o starannie dobrany stos technologiczny, który zapewnia:

    \subsection{Kluczowe zalety architektury}

    \begin{itemize}
        \item \textbf{Bezpieczeństwo} - wielowarstwowa ochrona danych finansowych
        \item \textbf{Wydajność} - optymalizacja na każdym poziomie stosu
        \item \textbf{Skalowalność} - możliwość łatwego rozszerzania funkcjonalności
        \item \textbf{Utrzymywalność} - czytelna architektura i dokumentacja
        \item \textbf{Developer Experience} - nowoczesne narzędzia deweloperskie
    \end{itemize}

    \subsection{Możliwości rozbudowy}

    Architektura systemu pozwala na łatwe dodanie:
    \begin{itemize}
        \item Mikroservices (podział monolitu)
        \item Message Queues (Redis/RabbitMQ)
        \item External APIs (płatności, weryfikacja)
        \item Mobile Applications (React Native)
        \item Advanced Analytics (data warehouse)
    \end{itemize}

    Wybrane technologie stanowią solidną podstawę dla dalszego rozwoju aplikacji i implementacji zaawansowanych funkcjonalności bankowych.

\end{document}
